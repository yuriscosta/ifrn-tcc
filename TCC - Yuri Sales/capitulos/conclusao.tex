\chapter{Considerações Finais}
\label{chap:conclusao}
Corretoras de criptomoedas existem há menos de dez anos e, devido à necessidade de consultar informações e automatizar tarefas em tempo real, algumas delas já possuem uma \textit{API} para que seja possível acessar dados públicos e privados.

Desenvolvedores encontram dificuldade de lidar com tais \textit{APIs} por diferentes motivos, entre eles: documentação mal escrita e mal detalhada, erros inesperados, inconsistência de dados, ausência de tratamento de erros, limitações, entre outros.

Este trabalho apresenta uma análise comparativa das \textit{RESTful APIs} de seis diferentes \textit{exchanges} de criptomoedas baseado em padrões e boas práticas de desenvolvimento, objetivando servir como base para outros desenvolvedores que atuam desenvolvendo soluções para o mercado de criptoativos.

A fim de atingir os objetivos da pesquisa, o desenvolvimento foi dividido em três partes: escolha das regras que serão utilizadas como parâmetros, análise individual dos padrões e do funcionamento das \textit{APIs} conforme cada critério e levantamento das pontuações gerais obtidas pelas corretoras.

Na primeira fase, um escopo para definir as regras foi delimitado. Foram escolhidos os critérios mais básicos e necessários que todas as aplicações devem seguir, levando em consideração o contexto que essas \textit{APIs} estão inseridas. 

Na análise, o trabalho foi realizado de forma individual, para que houvesse um aprofundamento no estudo e nos testes de cada
\textit{exchange}. Além disso, foi realizada uma pesquisa a respeito das limitações de cada \textit{API REST},
servindo, no fim, como um complemento à análise.

E, por último, um resumo de cada conjunto de regras por categoria, detalhando o cumprimento das mesmas e aprofundando na comparação e nas causas que podem levar ou não ao cumprimento de tais regras.

Durante este estudo, foi possível notar que as equipes de desenvolvimento das corretoras pouco se preocupam com as boas práticas e os padrões definidos para aplicações \textit{RESTful}. Podendo isso ser prejudicial aos consumidores, que precisam desse serviço disponibilizado para desenvolver seus sistemas, e, até mesmo, à própria plataforma da \textit{exchange}, com dados irreais e outros possíveis erros.

No balanço total, a Binance foi a corretora que mais cumpriu as regras, seguida da Livecoin e da Huobi. Já a Bittrex, a Gate.io e a Poloniex, cumpriram menos da metade das regras, dessa forma, mostrando que as suas \textit{APIs} estão imaturas no quesito padronização do desenvolvimento.

\section{Limitações}

A principal limitação encontrada nesta pesquisa foi a falta de informações disponibilizadas pelas documentações das \textit{APIs}. Em algumas \textit{exchanges} analisadas, detalhes sobre os possíveis tipos de respostas de cada \textit{endpoint}, também sobre os \textit{status codes} e mensagens internas são omitidos. Em uma corretora específica, a Gate.io, sequer são informados os limites de requisições e o tempo de bloqueio após tal limite ser ultrapassado, dessa forma, dificultando os testes e as simulações realizadas.


\section{Trabalhos futuros}

O desenvolvimento de \textit{APIs} passa por diversos estágios, entre eles: gerenciamento, modelagem, documentação, versionamento, testes e governança. Por ser uma área vasta e complexa, a pesquisa pode, futuramente, ser estendida analisando a documentação escrita pelas corretoras e realizando testes unitários, de performance e de segurança

O estudo das \textit{exchanges} pode ser levado a outro patamar com o acréscimo de mais \textit{APIs} e com a inserção de mais um objeto de análise, os \textit{Websockets}, os quais, diferentemente das \textit{RESTful APIs}, estão menos presentes nas corretoras.