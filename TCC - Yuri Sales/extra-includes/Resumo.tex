% Resumo
% \begin{center}
% 	{\Large{\textbf{\myThesis}}}
% \end{center}

% \vspace{1cm}

% \begin{flushright}
% 	Autor: \myName\\
% 	Orientador: \mySupervisorName
% \end{flushright}

% \vspace{1cm}

% \begin{center}
% 	\Large{\textsc{\textbf{Resumo}}}
% \end{center}

% \noindent 

% \noindent\textit{Palavras-chave}: 

% \setlength{\absparsep}{18pt} % ajusta o espaçamento dos parágrafos do resumo
\begin{resumo}
	Com a popularização do \textit{Bitcoin} e de outras criptomoedas, casas de câmbio \textit{online} vêm surgindo para que os usuários possam comprar, guardar, transacionar e operar no mercado suas moedas digitais. Devido a esse aumento na oferta de \textit{exchanges} e de moedas, é possível desenvolver estratégias de mercado para compra e venda em diferentes lugares entre diferentes criptomoedas; e uma forma de facilitar esse processo é por meio da automatização do mesmo.
Tal automatização pode ser concebida através da construção de aplicações que consomem as \textit{APIs} das casas de câmbio. Essas \textit{APIs} ainda carecem de consistência de informações, qualidade, materiais de apoio e são repletas de limitações. Devido a essa imaturidade, desenvolvedores encontram dificuldade em trabalhar com essas \textit{APIs}. Este trabalho tem como objetivo realizar uma análise comparativa entre seis corretoras, utilizando como parâmetro os padrões de desenvolvimento e as boas práticas de \textit{design} de \textit{APIs}, assim como a verificação de suas limitações, e documentar os resultados a fim de auxiliar outros desenvolvedores.

  %\vspace{\onelineskip}

  \noindent
  {Palavras-chave}: Modelagem de APIs. REST. Exchanges de Criptomoedas.
\end{resumo}